% Options for packages loaded elsewhere
\PassOptionsToPackage{unicode}{hyperref}
\PassOptionsToPackage{hyphens}{url}
%
\documentclass[
]{article}
\usepackage{amsmath,amssymb}
\usepackage{iftex}
\ifPDFTeX
  \usepackage[T1]{fontenc}
  \usepackage[utf8]{inputenc}
  \usepackage{textcomp} % provide euro and other symbols
\else % if luatex or xetex
  \usepackage{unicode-math} % this also loads fontspec
  \defaultfontfeatures{Scale=MatchLowercase}
  \defaultfontfeatures[\rmfamily]{Ligatures=TeX,Scale=1}
\fi
\usepackage{lmodern}
\ifPDFTeX\else
  % xetex/luatex font selection
\fi
% Use upquote if available, for straight quotes in verbatim environments
\IfFileExists{upquote.sty}{\usepackage{upquote}}{}
\IfFileExists{microtype.sty}{% use microtype if available
  \usepackage[]{microtype}
  \UseMicrotypeSet[protrusion]{basicmath} % disable protrusion for tt fonts
}{}
\makeatletter
\@ifundefined{KOMAClassName}{% if non-KOMA class
  \IfFileExists{parskip.sty}{%
    \usepackage{parskip}
  }{% else
    \setlength{\parindent}{0pt}
    \setlength{\parskip}{6pt plus 2pt minus 1pt}}
}{% if KOMA class
  \KOMAoptions{parskip=half}}
\makeatother
\usepackage{xcolor}
\usepackage[margin=1in]{geometry}
\usepackage{longtable,booktabs,array}
\usepackage{calc} % for calculating minipage widths
% Correct order of tables after \paragraph or \subparagraph
\usepackage{etoolbox}
\makeatletter
\patchcmd\longtable{\par}{\if@noskipsec\mbox{}\fi\par}{}{}
\makeatother
% Allow footnotes in longtable head/foot
\IfFileExists{footnotehyper.sty}{\usepackage{footnotehyper}}{\usepackage{footnote}}
\makesavenoteenv{longtable}
\usepackage{graphicx}
\makeatletter
\newsavebox\pandoc@box
\newcommand*\pandocbounded[1]{% scales image to fit in text height/width
  \sbox\pandoc@box{#1}%
  \Gscale@div\@tempa{\textheight}{\dimexpr\ht\pandoc@box+\dp\pandoc@box\relax}%
  \Gscale@div\@tempb{\linewidth}{\wd\pandoc@box}%
  \ifdim\@tempb\p@<\@tempa\p@\let\@tempa\@tempb\fi% select the smaller of both
  \ifdim\@tempa\p@<\p@\scalebox{\@tempa}{\usebox\pandoc@box}%
  \else\usebox{\pandoc@box}%
  \fi%
}
% Set default figure placement to htbp
\def\fps@figure{htbp}
\makeatother
\setlength{\emergencystretch}{3em} % prevent overfull lines
\providecommand{\tightlist}{%
  \setlength{\itemsep}{0pt}\setlength{\parskip}{0pt}}
\setcounter{secnumdepth}{-\maxdimen} % remove section numbering
\usepackage{bookmark}
\IfFileExists{xurl.sty}{\usepackage{xurl}}{} % add URL line breaks if available
\urlstyle{same}
\hypersetup{
  pdftitle={Final\_Paper},
  hidelinks,
  pdfcreator={LaTeX via pandoc}}

\title{Final\_Paper}
\author{}
\date{\vspace{-2.5em}}

\begin{document}
\maketitle

{
\setcounter{tocdepth}{2}
\tableofcontents
}
\textbf{This project investigates how the volume of photos attached to
Yelp reviews relates to a restaurant's average ratings, and if the type
of picture (menu, food \& drink, or environment) affects this
relationship.}

\section{1. Research Motivation}\label{research-motivation}

The influence of online reviews and electronic word of mouth (eWOM) on
consumer perceptions and decision-making is evident in today's digital
society (Wang et al., 2021). As such, this phenomenon has been widely
researched in marketing. The influence of reviews is especially
pronounced in the case of ``experience goods''. For example, restaurant
services are perceived as riskier to evaluate before purchase
(Weisskopf, 2018). Previous findings suggest that by reading restaurant
reviews, customers are able to reduce this perceived risk (Parikh et
al., 2014). Further supporting this claim, Luca (2016, p.3) demonstrates
that ``a one-star increase in Yelp rating leads to a 5--9 percent
increase in revenue''.

Photos have become a key component of online reviews as they add a
visual element that allows users to better communicate their opinions
(Li et al., 2021). However, while text-based reviews have been
extensively studied, the association between the number of photos
included in reviews and a restaurant's average rating remains
underexplored. Additionally, current literature is limited with regard
to how different types of photos may impact this relationship. To
address this gap in the literature, the following research question has
been formulated:

\section{2. Research Question}\label{research-question}

\textbf{How does the total number of photos included in Yelp reviews
influence a restaurant's average rating, and to what extent does the
type of photo (food, environment, menu) moderate this relationship?}

\section{3. Managerial Relevance}\label{managerial-relevance}

From a managerial and business perspective, the stakeholders, namely
restaurant owners and managers, may gain a deeper understanding from the
findings of this project on how to potentially raise ratings per
restaurant, reduce average rating volatility and increase visibility and
engagement. Luca (2016) concluded that the average rating of a
restaurant guides customers to make informed decisions. Ultimately, this
may lead to an improved reputation and subsequently increased customer
demand.

Furthermore, managers can enhance their marketing strategy by
strategically encouraging customers to publish reviews with photos on
platforms such as ``Yelp'' or ``Tripadvisor'' as it reduces customer
uncertainty. To encourage reviews that include photos, managers may add
QR codes on the menu or receipts that could incentivize customers with
rewards for the uploaded photos. Additional innovative approaches are
table toppers and Wi-Fi login prompts inviting the customers to share a
photo of the food or environment. By prompting customers to share
additional photos, managers may be more inclined to detect recurring
issues, creating a feedback loop.

\section{4. Data}\label{data}

\subsection{4.1 Data Sourcing}\label{data-sourcing}

This project uses two data sets obtained from the
\href{https://business.yelp.com/data/resources/open-dataset/}{Yelp Open
Dataset}. One data set contains business-related information, while the
other provides detailed photo data, including both the images and their
associated classifications, such as ``food,'' ``drink,'' ``menu,''
``inside,'' and ``outside.''

\subsection{4.2 Data Preparation and
Variables}\label{data-preparation-and-variables}

To answer the research question, these two data sets were merged,
cleaned, and transformed. This process resulted in
``final\_dataset.csv,'' which serves as the data set used for analysis.
This CSV file includes 29,374 observations across 10 variables. An
overview of these variables is provided below.

\begin{longtable}[]{@{}
  >{\raggedright\arraybackslash}p{(\linewidth - 4\tabcolsep) * \real{0.1413}}
  >{\raggedright\arraybackslash}p{(\linewidth - 4\tabcolsep) * \real{0.7391}}
  >{\raggedright\arraybackslash}p{(\linewidth - 4\tabcolsep) * \real{0.1196}}@{}}
\caption{Overview of Variables in Final Dataset}\tabularnewline
\toprule\noalign{}
\begin{minipage}[b]{\linewidth}\raggedright
Variable
\end{minipage} & \begin{minipage}[b]{\linewidth}\raggedright
Description
\end{minipage} & \begin{minipage}[b]{\linewidth}\raggedright
Data\_Class
\end{minipage} \\
\midrule\noalign{}
\endfirsthead
\toprule\noalign{}
\begin{minipage}[b]{\linewidth}\raggedright
Variable
\end{minipage} & \begin{minipage}[b]{\linewidth}\raggedright
Description
\end{minipage} & \begin{minipage}[b]{\linewidth}\raggedright
Data\_Class
\end{minipage} \\
\midrule\noalign{}
\endhead
\bottomrule\noalign{}
\endlastfoot
business\_id & The unique Yelp ID of the business & Character \\
name & The business name as shown on Yelp & Character \\
attributes & The map on Yelp of a restaurant's amenities, services, and
policies & List \\
categories & The list of Yelp categories of cuisines for the business &
Character \\
stars & The average Yelp scale star rating (1--5) & Numeric \\
review\_count & The total number of Yelp reviews & Numeric \\
environment & The number of environment photos & Numeric \\
food \& drink & The number of food \& drink photos & Numeric \\
menu & The number of menu photos & Numeric \\
\end{longtable}

\section{5. Data Exploration}\label{data-exploration}

Before conducting the analysis, it is essential to gain a comprehensive
understanding of the data set to identify its characteristics, patterns,
and potential limitations.

The data contains 29,374 restaurant observations, each with measures of
average Yelp rating (1--5 scale) and photo counts. These counts are
broken down by category (food \& drink, environment, and menu), and also
include a total photo count. The mean restaurant rating is approximately
3.74 stars, indicating a moderate overall quality perception. The
average total number of photos per restaurant is about 59. However, it
is important to note that the distribution is heavily right-skewed. This
means that many restaurants have only a few photos, while a small number
have hundreds. This uneven distribution is important context as it
indicates that although the large sample size guarantees statistical
precision the size of the effects should be considered realistically.

Additional data exploration and visualizations conducted using ggplot2
are documented in the data\_exploration file.

\section{6. Method}\label{method}

This project employs linear regression analysis as the primary research
method, chosen for its suitability in examining the relationship between
restaurant ratings and photo-related variables while allowing for
straightforward interpretation of effects.

To address the research question, a series of linear regression models
were estimated. First, a simple linear regression model was used to
obtain a baseline understanding of the relationship between the average
star rating of a restaurant and its total number of photos.

The baseline model is specified as follows:

\(Stars=β0​+β1​(Total Photos)+ε\)

Second, a multiple linear regression model was estimated. This second
model expands upon the baseline by including additional predictors, such
as photo category (food and drink, environment, and menu), and an
interaction term. The inclusion of an interaction term allows for an
assessment of whether the effect of photo quantity varies depending on
the dominant photo category. This modeling approach allows for a better
understanding of both the combined and conditional effects of photo
content on restaurant ratings.

As such, the main model is formally specified as follows:

\(Y=β0+β1(Photos)+β 2(Photo Category)+β3(Photos×Photo Category)+ε\)

where \(Y\) represents the average restaurant rating, \(Photos\) denotes
the total number of photos, and \(Photo Category\) indicates the
dominant photo type category. The interaction term
\(β3(Photos×Photo Category)\) captures whether the influence of photo
quantity on ratings differs across categories.

Lastly, an interaction model was estimated to more explicitly assess
whether the effect of photo volume on ratings varies by the predominant
type of photo. This model introduces interaction terms between the total
number of photos and the dominant photo category, allowing for testing
whether the slope of the relationship between total photos and star
ratings differs across food-, menu-, and environment-dominant
restaurants, rather than assuming a uniform effect.

The model is specified as follows:
\(Stars=β0​+β1​(Photos Centered)+β2​(Category Dominant)+β3​(Interaction)+ε\)

\section{7. Analysis}\label{analysis}

\subsection{7.1 Baseline Model}\label{baseline-model}

To establish a preliminary understanding of the relationship between the
total number of photos in reviews and the average star rating of
restaurants, a baseline linear regression model was estimated. The
output of the linear regression is presented below:

\begin{center}\includegraphics[width=0.8\linewidth]{../gen/output/main_effect} \end{center}

\subsubsection{Model Interpretation}\label{model-interpretation}

The coefficient for Total Photos is positive and highly significant
(\(β = 0.0097, p < .001\)), indicating that restaurants with more photos
tend to receive slightly higher average ratings. More specifically, each
additional photo uploaded to a restaurant's Yelp page is associated with
a 0.0097 point increase in its average star rating. In practical terms,
an increase of 50 photos corresponds to approximately +0.48 stars on the
1--5 rating scale.

\textbf{Explanatory Power:} The model's explanatory power is very low
(\(R² = 0.015\)), indicating that photo quantity alone explains only a
small fraction of the variation in ratings. Most of the variation in
ratings is driven by other factors (food quality, service, price,
location, etc.), not simply by the number of photos. Thus, while the
relationship is statistically overwhelming due to the sample size, the
actual effect in practice is quite small.

Despite this limited explanatory power, the positive relationship aligns
with theoretical expectations that photos in online reviews help
consumers better evaluate restaurants, reducing perceived risk and
increasing confidence in their decision-making, which in turn influences
average ratings.

\subsection{7.2 Main Model}\label{main-model}

To examine whether different types of photos contribute differently to
restaurant ratings, a multiple linear regression model was estimated.
The dependent variable remains the average star rating, while the
independent variables represent photo counts across distinct categories:
Food \& Drink, Menu, and Environment. As such, this model builds upon
the baseline results by decomposing the overall photo effect into more
specific photo types.

The output of the regression is presented below:

\includegraphics[width=1\linewidth]{../gen/output/model_categories}

\subsubsection{Model Interpretation:}\label{model-interpretation-1}

When controlling for all photo categories, menu photos exhibit the
strongest positive association with average ratings. Specifically, each
additional menu photo is associated with a 0.13-star increase, holding
the number of environment and food \& drink photos constant. Environment
photos also show a positive effect, though smaller in magnitude, with
each additional photo contributing approximately +0.022 stars. In
contrast, food \& drink photos do not have a statistically significant
effect, suggesting that once menu and environment photos are accounted
for, additional images of food provide little incremental predictive
value.

These results suggest different functional roles for each photo type: -
Menu photos likely reduce informational uncertainty, allowing consumers
to evaluate prices, portion sizes, and offerings before visiting, which
increases confidence and trust. - Environment photos convey restaurant
atmosphere and cleanliness, shaping emotional expectations. - Food
photos, though abundant in the data set, may suffer from saturation and
variable quality, explaining their negligible incremental impact.

\textbf{Explanatory Power:} The inclusion of photo types as moderators
increases the model's explanatory power to \(R² = 0.023\), representing
almost a 50\% improvement over the baseline model. While the overall
explained variance remains small, the findings from this model indicate
that photo type carries meaningful information beyond quantity.

\subsection{7.3 Interaction Model}\label{interaction-model}

To assess whether the effect of photo volume on ratings varies by the
predominant type of photo, an interaction model was estimated,
introducing interaction terms between the total number of photos and the
dominant photo category. This specification allows testing whether the
slope of the relationship between total photos and star ratings differs
across food-, menu-, and environment-dominant restaurants, rather than
assuming a uniform effect.

\includegraphics[width=1\linewidth]{../gen/output/model_central_moderation}

\subsubsection{Model Interpretation:}\label{model-interpretation-2}

For environment-dominant restaurants (the reference group) with an
average number of photos (5.8), the expected star rating is 3.74. Each
additional photo above the mean, for restaurants where environmental
photos is the dominant group, is associated with a small but
statistically significant increase in rating of 0.0082 (p \textless{}
0.001), consistent with the baseline finding that more photos slightly
raise ratings. The model further indicates that restaurants dominated by
food and drink photos have lower average ratings than
environment-dominant restaurants (\(−0.176 stars, p < 0.001\)), while
menu-dominant restaurants show a small positive difference (+0.165
stars), though this is not statistically significant (\(p = 0.130\)).

The interaction terms indicate that the effect of additional photos
differs by photo type. For food \& drink-dominant restaurants, each
extra photo contributes an additional 0.0027 to the star rating beyond
the effect for environment-dominant restaurants (\(p = 0.0096\)),
suggesting a slightly stronger positive relationship. For menu-dominant
restaurants, each extra photo adds 0.027 points compared to
environment-dominant restaurants, but this effect is not significant
\((p = 0.266\)), indicating that photo quantity does not meaningfully
alter ratings in this category.

\textbf{Explanatory Power:} The model explains approximately 2.6\% of
the variance in star ratings (\(R² = 0.026\)) and is statistically
significant. While modest, this shows that both the number and type of
photos contribute meaningfully to predicting ratings. The low R² also
highlights that many other factors, such as food quality, service, and
pricing, likely influence ratings beyond what is captured by visual
content.

\subsection{7.4 Overview of Findings and Managerial
Recommendations}\label{overview-of-findings-and-managerial-recommendations}

Across all three models, the findings consistently demonstrate that the
total number of photos on Yelp has a significant effect on average
restaurant ratings. The baseline model establishes that restaurants with
more photos tend to receive higher average ratings, supporting the idea
that visual information enhances perceived credibility and reduces
consumer uncertainty.

When investigating photo type, the results reveal that not all photos
contribute equally. Menu photos have the strongest positive relationship
with ratings, suggesting that they help customers form realistic
expectations about offerings and prices. Environment photos also
contribute positively, likely because they signal cleanliness,
atmosphere, and other attributes that shape expectations before a visit.
By contrast, food photos appear less influential once other types are
controlled for, possibly due to oversaturation or inconsistent quality.

The interaction model further refines these insights by showing that
while the overall photo--rating relationship is robust across
categories, the strength of the effect varies slightly depending on
which type of photo dominates a restaurant's gallery. Food-dominant
restaurants start with lower ratings but benefit more from increasing
photo volume, whereas menu-dominant restaurants start higher but do not
gain additional advantage from extra photos.

Taken together, the findings suggest that managers should can implement
strategies to improve perceived reputation and attract potential
customers. Practical steps include: - Encouraging customers to upload
diverse photos, especially of menus and interiors. - Prompting visual
engagement through in-store QR codes, receipts, or social media
campaigns. - Monitoring photo composition over time to maintain a
balanced gallery that aligns with brand positioning.

\section{8. Conclusion}\label{conclusion}

This project investigated the relationship between the number and type
of photos in Yelp reviews and restaurants' average star ratings. The
analysis demonstrates that photos play a meaningful role in influencing
ratings, but their impact varies by content category. Menu and
environment photos are most influential, while food and drink photos
alone have limited effect.

From a managerial perspective, these findings provide actionable
guidance for restaurants seeking to enhance their online reputation.
Encouraging customers to upload specific types of photos through
incentives or prompts can increase perceived quality, foster trust, and
ultimately improve ratings. For review platforms such as Yelp,
incorporating photo type alongside quantity into recommendation
algorithms could improve the accuracy and relevance of suggestions for
users.

Overall, this study highlights the value of user-generated visual
content in online reviews and its role in shaping consumer
decision-making, providing both practical insights for businesses and a
foundation for further research on visual eWOM in the restaurant
industry.

\section{9. Limitations and Future
Research}\label{limitations-and-future-research}

While the models identify significant relationships between photo
characteristics and Yelp ratings, several limitations remain.The
analysis is correlational, not causal, which means restaurants with
higher ratings may simply attract more photos. Moreover, unobserved
factors such as location, price level, cuisine, and review text
sentiment were not controlled for, potentially biasing estimates.
Lastly, the linear specification also assumes constant effects across
photo volumes, which may overlook diminishing returns. Future research
could incorporate elements such as photo quality,or sentiment to better
capture how photos truly shape consumer perceptions over time.

\section{References:}\label{references}

Li, C., Kwok, L., Xie, K. L., Liu, J., \& Ye, Q. (2021). Let Photos
Speak: The Effect of User-Generated Visual Content on Hotel Review
Helpfulness. Journal of Hospitality \& Tourism Research, 47(4),
109634802110191. \url{https://doi.org/10.1177/10963480211019113}

Luca, M. (2016). Reviews, Reputation, and Revenue: The Case of Yelp.com.
Harvard Business School NOM Unit Working Paper, 12(016).
\url{https://doi.org/10.2139/ssrn.1928601}

Parikh, A., Behnke, C., Vorvoreanu, M., Almanza, B., \& Nelson, D.
(2014). Motives for reading and articulating user-generated restaurant
reviews on Yelp.com. Journal of Hospitality and Tourism Technology,
5(2), 160--176. \url{https://doi.org/10.1108/jhtt-04-2013-0011}

Wang, Y., Kim, J., \& Kim, J. (2021). The financial impact of online
customer reviews in the restaurant industry: A moderating effect of
brand equity. International Journal of Hospitality Management, 95,
102895. \url{https://doi.org/10.1016/j.ijhm.2021.102895}

Weisskopf, D. J.-P. (2018, September 30). Online Customer Reviews: Their
Impact on Restaurants. Hospitalityinsights.ehl.edu.
\url{https://hospitalityinsights.ehl.edu/online-customer-reviews-restaurants}

\end{document}
